\setlength{\LSTxleftmargin}{2ex}%
\setlength{\LSTxrightmargin}{0ex}%
\setcounter{LSTcounterCharsPerLineOfCode}{79}%
%
\makeatletter%
\setlength{\LSTbasewidth@tmp}{\textwidth}%
\addtolength{\LSTbasewidth@tmp}{-\LSTxleftmargin}%                   xleftmargin
\addtolength{\LSTbasewidth@tmp}{-\LSTxrightmargin-\LSTxleftmargin}% compensation
\addtolength{\LSTbasewidth@tmp}{0.75ex}%        approx width of a-z in this font
\setlength{\LSTbasewidth}{\LSTbasewidth@tmp/\theLSTcounterCharsPerLineOfCode}%
\makeatother%
%
\lstset{%
xleftmargin=\LSTxleftmargin,%
xrightmargin=\LSTxrightmargin,%
basewidth=\LSTbasewidth%
}%
%
\colorlet{LSTcolorKeyword}{orange}%                 for, if, end, continue, etc.
%
\begin{lstPython}
#-------------------------------------------------------------------------------
"""
@Purpose : Syntax coloring of Python code.
@Author  : Sigve Karolius and Tore Haug-Warberg
@Address : Department of Chemical Engineering, NTNU, Norway
@Source  : Python 2.7 
@Date    : 2015
@License : Free, but please feel free to acknowledge us...
"""
#-------------------------------------------------------------------------------

def                                                              # reserved word
type()                                                         # global function
(1+2/3-4)*5,[6,7,8e9]                                             # basic tokens
e123.pop()                                              # object.instance_method
self                                                           # pseudo-variable
False                                                         # program constant
dict.__str__()                               # class_name.__attribute_method__()
dict.__doc__                                          # class_name.__attribute__
math.cos()                                                     # module.function
'str'                                                                   # string

from numpy.linalg import solve                    # global functions from module
from math import log, e                           # global functions from module

fil  = './Foo/bar/tmp.txt'                                     # file descriptor
file = open(fil, 'w')            # file is a class, don't use for variable names

with open(fil) as f:
  for line in f:
    print type(line)                                           # global function

fun = lambda : [[gas['p']-liq['p']], [gas['mu']-liq['mu']]]    # lambda function
jac = [[1 if not abs(i-j)-1 else -2 if not j-i else 0 for i in ri] for j in rj]

class PhaseState(object):
   """Thermodynamic state calculator deriving from class object"""

  def __call__(self):
    self.state['p'] = 8*t*r/(3-r) - 3*r**2
    self.state['mu'] = t*log( 8*t*r/(3-r) ) + t*r/(3-r) - 9*r/4

  def __str__(self, outfile=None, props=[]):
    props.append('mu')
    props.sort()
    tup = tuple([self.state[prop] for prop in props])
    if outfile: outfile.write('%8.6f\n' % tup)
    return None

gas = PhaseState(t, 0.1)           # create object 'gas' of the PhaseState class

print(gas.__doc__)          # display class (or function, or variable) docstring
str(gas,fil)                                 # display output from gas.__str__()
dir(gas)             # display a list of attributes specific to the 'gas' object

for ti in [0.7, 0.8, 0.9, 0.95, 0.97, 0.99, 0.999, 0.9999]:
  converged = False
  while not converged:
    dx = -solve(jac, fun())
    x += dx
    converged = max(abs(dx)) < 1e-10  # conflict between exponent & function 'e'

#-------------------------------------------------------------------------------
#2345678901234567890123456789012345678901234567890123456789012345678901234567890
#-------------------------------------------------------------------------------
\end{lstPython}
